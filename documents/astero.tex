\documentclass[12pt, preprint]{aastex}

\input{vc}
\newcounter{affil}
\newcommand{\project}[1]{\textsl{#1}}
\newcommand{\Kepler}{\project{Kepler}}
\newcommand{\given}{\,|\,}
\newcommand{\DeltaTKepler}{\Delta T_{\mathrm{Kep}}}

\begin{document}

\title{
  Asteroseismic timing for exoplanet discovery and characterization
  \\ or \\
  Stellar oscillation modes as clocks for exoplanet discovery
}
\author{
  David~W.~Hogg\altaffilmark{\ref{ccpp},\ref{cds},\ref{mpia},\ref{email}},
  and
  Dan~Foreman-Mackey\altaffilmark{\ref{ccpp}}
}
\date{DRAFT VERSION / \texttt{\githash}\ / \today\ / NOT FOR DISTRIBUTION}
\refstepcounter{affil}\label{ccpp}\altaffiltext{\ref{ccpp}}{%
CCPP, NYU Physics
}
\refstepcounter{affil}\label{cds}\altaffiltext{\ref{cds}}{%
NYU CDS
}
\refstepcounter{affil}\label{mpia}\altaffiltext{\ref{mpia}}{%
MPIA
}
\refstepcounter{affil}\label{email}\altaffiltext{\ref{email}}{%
david.hogg@nyu.edu
}

\begin{abstract}
A large fraction of stars have highly coherent (quality factors of
thousands or greater) stellar oscillations in multiple modes.
In particular, a periodogram of \Kepler\ ordinary-cadence data on a
red giant star routinely exhibits a dozen or so unresolved, high
signal-to-noise oscillation modes.
These modes can be used like clocks, sensitive to anomalous time
delays generated by stellar motions.
In particular, they can be used to detect and characterize the
movement of a star in response to orbital companions, including
exoplanets.
Here we discuss the opportunities for using timing residuals in
stellar oscillation modes to discover and characterize exoplanet
companions.
We implement and demonstrate a method on real \Kepler\ data.
We find that we can detect exoplanet-induced timing residuals with
amplitudes of $\sim XXX$~s in some typical red-giant lightcurves.
For context, Jupiter induces timing residuals in the Sun with an
amplitude of 2.3~s.
\end{abstract}

\section{Introduction}

...Pulsar planets.  Exactly how do you find these?

...Stellar oscillations.  They are freakin' coherent.  There are many
modes.  They are high in signal-to-noise in the \Kepler\ data.

...Time delay as a function of orbital phase.  Different from radial
velocity, similar to astrometric wobble.  Encodes as much information
about the orbit as the radial velocity, but with different units.

...Context of the Solar System and known Kepler and RV planets.

...Connection to R\o mer and history.

\section{Information in the data}

If you can measure the timing amplitude $T_0$, you get a measure of
the line-of-sight semi-major axis projection $a\,\sin i$.
The amplitude $T_0$ rises linearly with the separation $a$ between the
planet and star, just like the astrometric wobble amplitude.
If you can \emph{also} measure the astrometric amplitude $\theta_0$,
you get the distance.
The proper-motion distance, to be very specific.
If you measure the timing amplitude $T_0$ and also transits, the
inclination is usually very highly constrained and $a\,\sin i$ becomes
a complete measurement of $a$.
(If you can also measure the radial velocity and period, you get a
measure of the speed of light.)

In principle, the full Keplerian eccentric orbit is encoded in the
time history, so in principle we can measure the eccentricity and
orientation of the argument of perihelion, up to a projection (on to
the line of sight).
The time-delay function of time is just the integral of the radial
velocity function of time; anything that can be inferred from
radial-velocity data can be inferred from time-delay data.
That said, the sensitivities to noise will be different, and good
inferences will require very good time-delay data.

Given a set of oscillation modes $m$ with oscillation periods $P_m$,
our ability to find a long-period signal (meaning a signal which
varies on time-scales much longer than the oscillation frequencies) of
known functional form and amplitude $T_0$ (time units) is given
roughly by:
\begin{eqnarray}
\frac{1}{\sigma_T^2} &\sim& \sum_m \frac{\DeltaTKepler^2}{P_m^4}\,\left(\frac{S}{N}\right)_m^2
\quad ,
\end{eqnarray}
where $\sigma_T$ is the uncertainty on the amplitude $T_0$,
$\DeltaTKepler$ is the lifetime of the \Kepler\ mission data, and
$(S/N)_m$ is the signal-to-noise with which mode $m$ is detected.
This formula assumes that the oscillations are more-or-less coherent
over the lifetime of the mission, which appears to be true
empirically.
There is a pre-factor of order unity that depends on the functional
form of the signal and other details.
This precision estimate depends incredibly strongly on period:
Much shorter periods are far more valuable.
Hence the value of the millisecond pulsars!

\section{Methods}

Our general approach is to take periodograms of the data.
The idea is that if we distort the time axis of the data---effectively
``un-doing'' the time-delay effects created by the orbit of a
planetary companion---the periodogram will get more informative as the
distortion better matches the inverse of that induced by the planet.

The method is based on the important fact that a periodogram is
related (beautifully) to a likelihood function (\citealt{bretthorst}).
\begin{eqnarray}
\ln p(d\given P_m) &=& \frac{C(P_m)}{\sigma^2} + \mbox{const}
\quad,
\end{eqnarray}
where $d$ is the light-curve data, $P_m$ is the period of a putative
sinusoidal signal---here an asteroseismology mode $m$---generating the
data, $C(P_m)$ is the periodogram (best-fit squared amplitude as a
function of period $P_m$) and $\sigma^2$ is the per-datum noise
variance (assumed Gaussian, independent, and all identical).

...define $C(P)$ here.

Of course we are not really looking for the modes at periods $P_m$;
for our purposes, asteroseismology itself is a ``solved problem''.
We are looking for a time-delay function $h(t\given \omega)$ such that
if we un-delay the data by this function, each asteroseismology mode
$m$ becomes more faithfully represented in the data.
Here $\omega$ is the set of exoplanet parameters, which include a
time-delay amplitude $T_0$, a period $P$, an eccentricity $e$, an
argument of perhielion $\varpi$ and so on.

...define $C(P\given \omega)$ here.

...describe finding a peak and measuring it as a function of $\omega$.

...Information coming from multiple modes can be combined.  Describe
this.

...Approximations abound.

\section{Experiments with \Kepler\ data}

\section{Discussion}

...Key summary

...Possibility of characterizing all known exoplanets hosted by giant stars

...Possibility of search

\acknowledgements
It is a pleasure to thank
  Hans-Walter Rix (MPIA)
for useful discussions.
This research was partially supported by the NSF (grant XXX), NASA
(grant XXX), and the Moore and Sloan Foundations.
This research made use of the NASA Astrophysics Data System, and
open-source software including Python, numpy, scipy, and matplotlib.

\begin{thebibliography}{}\raggedright
\bibitem[Bretthorst(1988)]{bretthorst}
  Bretthorst, G. L., 1988,
  \textit{Bayesian Spectrum Analysis and Parameter Estimation},
  in \textit{Lecture Notes in Statistics} \textbf{48},
  Springer-Verlag, New York
  \footnotesize{(\url{http://bayes.wustl.edu/glb/book.pdf})}
\end{thebibliography}

\end{document}
